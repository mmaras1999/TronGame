\documentclass[11pt]{article}
\usepackage[utf8]{inputenc}
\usepackage[T1]{fontenc}
\usepackage{mathpazo}
\usepackage{titlesec}
\usepackage{indentfirst}
\usepackage{geometry}

\titleformat{\section}
  {\Large\bfseries}{\thesection}{2em}{}

\begin{document}
	\newgeometry{tmargin=3cm, bmargin=3cm, lmargin=3.5cm, rmargin=3.5cm}

	\begin{titlepage}
		\newcommand{\HRule}{\rule{\linewidth}{0.5mm}}
		\center

		\textsc{\LARGE Uniwersytet Wrocławski}\\[1.5cm]
		\textsc{\Large Wstęp do programowania w języku C}\\[0.5cm]
		\textsc{\large Projekt}\\[0.5cm]

		\HRule\\[0.4cm]
		{\huge\bfseries Tron}\\[0.4cm]
		\HRule\\[1.5cm]

		\large
		\textit{Autor}\\
		Michał Maras

		\vfill\vfill\vfill\vfill

		{\large 7 lutego 2019}

	\end{titlepage}

	\newpage
	
	\section{Zasady gry}
		Tron to gra przeznaczona dla dwóch graczy. Na samym początku są oni rozstawieni symetrycznie na planszy. Gracze poruszają się na zmianę. W każdej kolejce muszą się przemieścić na jedno z sąsiednich pól. W momencie przesunięcia się na dane pole zostaje ono zablokowane do końca gry. Przegrywa ten gracz, który jako pierwszy przemieści się na zablokowane wcześniej pole.
	
	\section{Opis programu}

		\subsection{Menu główne}
		Po uruchomieniu programu użytkownik zobaczy menu główne. Może on kliknąć w różne przyciski za pomocą myszy.
		\noindent Przycisk "Exit" wychodzi z programu. Natomiast po zaznaczeniu opcji "Start Game" gra pozwoli nam wybrać ustawienia naszej rozgrywki. Możemy tu wybrać:
		\begin{itemize}
  			\item czy dany gracz będzie sterowany przez użytkownika, czy sztuczną inteligencję
  			\item klawisze sterowania w przypadku użytkownika lub poziom trudności sztucznej inteligencji
 			\item kolory graczy
 			\item ilość wygranych rund koniecznych do wygrania rozgrywki
		\end{itemize}

		\noindent Po kliknięciu przycisku "Play" program przenosi użytkownika do właściwej rozgrywki.
		\\
		\\
		
		\subsection{Rozgrywka}
		W przypadku opcji sterowania przez sztuczną inteligencję dany gracz będzie sam wykonywał ruchy. W przeciwnym przypadku gracz będzie zawsze poruszał się na wprost. Po kliknięciu odpowiedniego przycisku przez użytkownika (strzałki w lewo lub w prawo, albo klawisze A lub D), gracz skręci we właściwą stronę. 

	\section{Implementacja}

		Program podzielony jest na dwie sceny - scenę Main Menu oraz Game Screen. Main Menu odpowiada za wybór ustawień gry i włączenie nowej rozgrywki. Game Screen za to zajmuje się już samą grą.

		\subsection{defines.h}
		\noindent W tym pliku są wszystkie makra często używane w programie (np. wielkość planszy, rozdzielczość ekranu, kolory, itp.)

		\subsection{main.c}
		\noindent Znajdują się tu trzy funkcje:

		\begin{itemize}
			\item main() -- wywołuje funkcje CreateWindow(), StartApp() (funkcja z application.c) oraz DestroyWindow()
			\item CreateWindow() -- tworzy i ustawia opcje okna
			\item DestroyWindow() -- zwalnia pamięć zajętą przez okno
		\end{itemize}

		\subsection{application.c / application.h}
		\noindent Znajduje się tu jedna funkcja - StartApp(), odpowiadająca za główną pętlę programu. Inicjalizuje ona scenę Main Menu, a następnie odpowiada za ciągłe odświeżanie logiki obecnej sceny oraz jej rysowanie na ekranie. Jeśli w pewnym momencie okno programu zostanie zamknięte, pętla się przerywa.

		\subsection{screens.c / screens.h}
		\noindent W tych plikach znajduje się deklaracja struktury sceny. Posiada ona m.in. Game Manager'a (odpowiadającego głównie za ustawienia rozgrywki i rozmowę z użytkownikiem), wszystkie potrzebne zasoby (czcionki, tekstury, itp.) oraz UI (interfejs użytkownika - tekst, obrazki i przyciski). Zawiera też odpowiednie funkcje (a właściwie wskaźniki na funkcje w zależności od rodzaju sceny) odświeżające logikę danej sceny, rysujące ją na ekranie i zamykające ją.
		\\
		\\
		\noindent Znajdują się tu funkcje:

		\begin{itemize}
			\item CreateNewScreen() -- funkcja tworząca nową scenę i ustawiająca ją domyślnie na Main Menu.
			\item ChangeScreen() -- funkcja zmieniająca obecną scenę na inną lub zamykająca ją. Podczas tego zwalniana jest również pamięć po wszystkich rzeczach używanych tylko w obecnej scenie - wszystkie tekstury, teksty, przyciski, itp. zostają zniszczone. W ten sposób zapobiegane są wszelkie wycieki pamięci.
			\item CheckLoadedResources() -- funkcja sprawdzająca, czy wszystkie zasoby sceny zostały poprawnie stworzone.
			\item DestroyResources() -- funkcja usuwająca wszystkie zasoby sceny
		\end{itemize}

		\subsection{UI.c / UI.h}
		\noindent Te pliki odpowiadają za interfejs użytkownika. Całość interfejsu w każdej ze scen podzielona jest na warstwy. Każda z nich zawiera odpowiednie tekstury, tekst i przyciski. Dzięki temu możemy szybko chować lub pokazywać daną warstwę, np. w szybki i łatwy sposób ukryć i wyłączyć wszystkie przyciski znajdujące się na ekranie. Jest to szczególnie istotne w przypadku warstw, które powinny pojawić się tylko po jakimś konkretnym wydarzeniu. Przykładem takiej warstwy jest wyskakująca na samym końcu rozgrywki informacja o zwycięzcy oraz przycisk do powrotu do menu głównego.

		\begin{itemize}
			\item struktura UI layer -- jedna warstwa - zawiera wszystkie przyciski, tekst i tekstury do niej należące
			\item funkcja DestroyLayer() -- zwalnia pamięć po danej warstwie
			\item struktura UI button -- przycisk - zawiera informacje o swoim położeniu oraz czy użytkownik obecnie na niego najeżdża lub klika
			\item struktura UI image -- obrazek - zawiera obraz oraz swoją pozycję
			\item struktura UI text -- tekst - zawiera informacje o swoim położeniu na ekranie, tekst, który powinien wyświetlać oraz czcionkę i kolor.
		\end{itemize}

		\subsection{board.h}
		\noindent Zawiera strukturę Board reprezentującą planszę gry.

		\subsection{gameManager.c / gameManager.h}
		\noindent Pliki te zawierają strukture Game Manager'a i wszystkie funkcje z nią związane. Game Manager posiada wszystkie informacje o grze (planszy, graczach, wyniku, obecnej rundzie, itp.). Ponadto zajmuje się ustawieniem i działaniem samej rozgrywki oraz wymianie informacji z użytkownikiem (pobieranie wejścia).
		\\
		\\
		\noindent Funkcje w tych plikach to: 

		\begin{itemize}
			\item ResetGame() -- resetuje grę - ustawia na nowo planszę, generuje nowe pozycje początkowe graczy, itp.
			\item UpdateGame() -- odświeża grę - uruchamia funkcje CheckInput(), MakeTurn(), oraz sprawdza warunki zakończenia gry
			\item CheckInput() -- sprawdza, czy użytkownik wcisnął przycisk odpowiadający za skręcanie
			\item MakeTurn() -- wykonuje turę gracza - odpowiednio przesuwa gracza, jeśli uderzy on w ścianę to zwraca informację o przegranej rundzie
		\end{itemize}

		\subsection{players.c / players.h}
		\noindent Zawierają strukturę player reprezentującą gracza. Posiada ona informacje: czy dany gracz jest sterowany przez sztuczną inteligencję czy użytkownika, które przyciski odpowiadają za skręcanie gracza, obecne jego położenie na planszy, następny ruch, który gracz zamierza wykonać, poziom trudności sztucznej inteligencji oraz kolor gracza.
		\\
		\\
		\noindent Posiada również trzy funkcje odpowiadające za poruszanie się gracza - sterowanie użytkownika (getHumanInput()), łatwej sztucznej inteligencji (getAIInputEasy()) i trudnej sztucznej inteligencji (getAIInputHard()).

		\subsection{AI.c / AI.h}
		\noindent Sztuczna inteligencja w grze jest podzielona na dwa rodzaje - łatwą i trudną. Łatwa polega na, jeśli jest to możliwe, zbliżaniu się do przeciwnika.
		\\
		\\
		\noindent W przypadku trudnej następny ruch przewidywany jest w nieco bardziej skomplikowany sposób. Używany jest do tego rekurencyjny algorytm minimax. Generuje on drzewo wszystkich możliwych stanów w grze do kilku tur do przodu. Dla każdego z liści drzewa wyznacza odpowiednią wagę (im ruch jest bardziej korzystny dla gracza obecnie wykonującego ruch, tym waga jest większa). Następnie dla każdego wierzchołka nie będącego liściem, jeśli następny ruch w tym stanie gry należy do gracza sterowanego sztuczną inteligencją to wybierana jest wartość największa spośród jego dzieci. W przeciwnym przypadku wybierana jest wartość najmniejsza. Gracz wykonuje ruch odpowiadający największej wartości wybranej przez korzeń drzewa. Do wyznaczenia wagi danego stanu używana jest funkcja heurystyczna. Za pomocą algorytmu BFS rozdziela ona planszę na dwie części - pola, do których obecny gracz dostanie się przed przeciwnikiem i na odwrót. Waga danego stanu to $(200 * (V_m - V_o) + 55 * (E_m - E_o))$, gdzie $V_m, V_o$ to ilość pól, do których szybciej dotrze kolejno gracz wykonujący ruch i przeciwnik, a $E_m$, $E_o$ to ilość dostępnych krawędzi znajdujących się w części gracza wykonującego ruch i przeciwnika.
		\\
		\\
		W tych plikach znajdują się:

		\begin{itemize}
			\item struktury bfsQuery, Node i Queue oraz funkcje CreateNewQuery(), pushQueue() i popQueue() -- struktury i funkcje pomocnicze do algorytmu BFS w celu wyznaczenia wagi stanu
			\item minimax() -- rekurencyjna funkcja minimax opisana wyżej
		\end{itemize}

		\subsection{MainMenu.c / MainMenu.h}
		\noindent W tych plikach znajdują się wszystkie funkcje inicjalizujące, odświeżające i rysujące warstwy UI wykorzystywane przez scenę Main Menu oraz funkcje tworzącą, aktualizującą, rysującą oraz niszczącą tą scenę.

		\subsection{GameScreen.c / GameScreen.h}
		\noindent W tych plikach znajdują się wszystkie funkcje inicjalizujące, odświeżające i rysujące warstwy UI wykorzystywane przez scenę Game Screen oraz funkcje tworzącą, aktualizującą, rysującą oraz niszczącą tą scenę.



\end{document}

